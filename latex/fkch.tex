\documentclass[nofonts, a4paper, oneside, 12pt]{article}

\usepackage{geometry}
\usepackage{xeCJK}
\usepackage{setspace}

\geometry{ left = 3cm, right = 3cm, bottom = 2cm, top = 3cm }
\setCJKmainfont[Scale=0.9, BoldFont=WenQuanYi Zen Hei]{AR PLBaosong2GBK Light}
\setCJKsansfont[Scale=0.9]{WenQuanYi Zen Hei}
\setCJKmonofont[Scale=0.9]{AR PL UMing CN}
\setCJKfamilyfont{kai}{AR PL UKai CN}
\setmainfont{FreeSerif}
\setsansfont{FreeSans}
\setmonofont{FreeMono}

\begin{document}
\onehalfspacing
\begin{center}
    \Large{以经济和生态建设为中心, 促进经济又好又快发展}
\end{center}

\begin{center}
\CJKfamily{kai}\small
No.121  姓名: 吴周辉\ \ 学号: 201428013229051
\end{center}

随着中国经济的不断发展, 生态问题也日益凸显出来. 过去以经济建设为中心,
确实推动了国民经济的快速发展, 人民生活也日益富裕起来. 但也有一些地方, 一些
领域没有处理好经济发展与生态建设的关系, 单纯地以经济建设为唯一目标, 以破坏
环境为代价, 导致能源资源与生态环境问题日益突出. 现在, 我国资源和环境的承受
力越来越小, 接近极限, 已经难以支撑 "高投入, 高排放, 高消耗, 低效率"
的粗放型增长, 一些大型城市已经出现了严重的 "城市病", 一些生态脆弱的区域
甚至已经不适宜人类的生产生活. 除此之外, 草原退货沙化, 地面沉降, 水土流失
等问题越来越突出, 迫切要求我国的经济增长方式实现改变.

在去年的 APEC 会议期间, 北京难得的好天气, 被人们称为 "APEC 蓝", 这其实是在
告诫我们, 好天气不能仅仅因为政治力量而得以出现, 而必须成为常态. 回顾党的
十八大以来, 习近平总书记多次对生态文明建设作出重要指示, 站在中国特色社会
主义的角度, 充分, 根本地阐述了我国目前面临的经济与生态之间的矛盾, 这个矛盾
要求我们必须马上对经济的发展方式作出大刀阔斧的改革.

建设生态文明, 本质上是选择发展方式的问题, 是用什么方法, 靠什么途径实现发展,
做大蛋糕的问题; 建设生态文明, 不仅仅是防治污染, 也不是不要发展, 而是用绿色,
循环, 低碳的方式实现发展, 从而在源头上减少污染物的产生和排放. 通过这条环境
友好型的发展之路, 才能实现由 "环境换取增长" 向 "环境优化增长" 的转变, 真正
做到经济建设与生态建设同步推进, 产业竞争力与环境竞争力一起提升, 物质文明
与生态文明共同发展, 才能培育好 "金山银山" 成为新的经济增长点, 又保护好 
"绿水青山", 在生态建设方面取得新的进展.

习总书记指出, 只有实行最严格的制度, 最严密的法治, 才能切实保障生态文明的
建设. 在党的第十八届四中全会上也明确提出, 采用严厉的法律来保障生态文明的
建设. 所有的这些都表明, 只有法律才能切实地保障生态文明的建设顺利进行.

目前仍然后有许多地方政府没有把生态文明建设放在重要位置, 在笔者看来, 主要
原因有以下两点:

一是某些地方官员的执政理念扭曲. 所谓的执政理念扭曲指的是少数地方基层领导
片面地认为只有把地方的经济搞上去了, 他们才能爬地更高, 才能当大官. 他们 
认为经济建设是他们升官发财的唯一道路, 只要把经济建设搞上去了, 则是 "一俊
遮百丑". 因此他们把主要的精力都放在了经济建设上, 而忽视了生态建设.

二是依法监管工作不到位. 虽然已经出台了大量的法律法规, 对生态建设作出了一
系列的规定, 但仍然有许多企业我行我素, 而地方政府也常常因为他们是纳税大户
而对他们睁一只眼, 闭一只眼. 法律法规流于形式, 无法落实. 即使偶有企业因
污染而受到处罚, 但处罚措施对这些企业来说无关痛痒. 违法成为过低, 也是生态
建设长期得不到有效发展的重要原因之一.

回顾世界历史, 许多国家都经历了先污染, 后治理的路子, 中国绝不能起他人的老
路. 事实证明, 在发展中把环境破坏了, 再补回去的成本可能比当初创造的财富还
要多, 更何况有些地方的重金属污染, 水污染, 土壤污染更是积重难返, 难以恢复.
这些教训是极为深刻的.

我国作为一个能源, 人口大国, 环境污染问题非常严峻. 要实现四个现代化, 我国 
必须走出一条与发达国家不同的路子出来, 否则只会误入歧途. 我们只有更加重视
生态环境这一生产力要素, 更加尊重自然生态的发展规律, 利用和保护好环境, 才
能更好地发展生产力, 人民才能生活地更加舒适, 更加幸福. 为了更好地促进生态
文件建设, 我们必须始终坚持科学发展观, 以全面, 科学地观点发展经济, 保护
环境. 

党的十八大四中会不会提出, 建立健全自然资源产权法律制度, 完善国土空间开发
保护方面的法律制度, 制定完善生态补偿和土壤, 水, 大气污染防治及海洋生态
环境保护等法律法规, 全力促进生态文明建设. 我们应当看到, 生态文明建设是一个
系统工程, 环环相扣, 缺了哪一个都不行, 是一块涉及生产方式, 生活方式, 思维
方式和价值观念的革命性变革, 按照系统工程的思路抓好生态文明建设重点任务的
落实, 除了建立完善相关法律法规, 还要完善经济社会发展考核评价体系, 切实把
资源消耗, 环境损害, 生态效益等指标纳入进去. 使之成为推进生态文明建设的重要
导向和约束, 对那些不顾生态环境盲目决策, 导致严重后果的领导干部也要追究责任.
同时还要加强生态文明宣传教育, 增强全民节约意识, 环保意识, 生态意识, 营造
爱护生态环境的良好社会风气.

我们每个人都可以为生态文明建设贡献自己的一份力量. 不乱扔垃圾, 尽量乘坐
公共交通工具, 使用节能家具. 生活中到处都是机会, 我们每个人都会影响周围的
环境, 只要我们人人都献出自己的一份力量, 这些力量汇集到一处, 将会非常的
强大.
\end{document}
